%----------------------------------------------------------------------------------------
\documentclass[twoside]{article}
\usepackage{amsmath}
\usepackage{scrextend}
\usepackage[hmarginratio=1:1,top=32mm,columnsep=20pt]{geometry} % margins
\usepackage{graphicx}
\usepackage{subfig}

% change quotes to american style double quotes
\usepackage [english]{babel}
\usepackage [autostyle, english = american]{csquotes}
\MakeOuterQuote{"}

\begin{document}
\linespread{1.3}
\title{Estimating size and distribution of pathogenic targets in rare disease}
\author{}
\date{September 2017}
\maketitle

% citation example: \cite{Langfelder2008a}
% figure ref example: ~\ref{fig:Dendrogram} with label tag inside the figure as \label{fig:Dendrogram}
% table ref example: ~\ref{table:GeneSet} with label tag inside the table as \label{tab:GeneSet}
% use `text in quotes' to get quotes facing the correct direction rather than 'text in quotes'

\section{Model and example using LOF from McRae et. al 2017}

\huge
$$E(m) = (1-\phi)N\mu + (1-\rho)N\mu + \pi N \frac{\mu \phi}{1-e^{-\mu \phi}}$$
\large
$E(m)$ is the expected number of mutations (or variants)
$\mu$ is the mutation rate of the target of interest (e.g. all loss of function mutations in genes)\\
$\phi$ is the proportion of the target that is pathogenic (e.g. proportion of haploinsufficient genes)\\
$\rho$ is the penetrance.
$\pi$ is the proportion of cases in the population that carry a pathogenic mutation in the target considered (e.g. ~14-20\% of DDD patients carry a pathogenic LOF)\\
$N$ is the size of the population\\

\normalsize
To explain the last bit: $\mu \phi$ is the expected number of mutations per person, and $1-e^{-\mu \phi}$ is the Poisson probability of that person having at least one mutation. So what we're interested in is what's the average number of mutations that a person with \emph{at least one} needs to carry to make the expectation correct. Or put another way, take the total number of mutations and split between people with zero or more than zero: $\mu \phi = 0\cdot e^{-\mu} + x \cdot (1-e^{-\mu})$, and solve for $x$. Note that for likely values of $\mu \phi$ for your project, that adjustment term is $\approx 1.0001$! 

In a high penetrance model, $\rho$ is close to 1 and the model is approximately equal to:

\huge
$$E(m) = (1-\phi)N\mu + \pi N $$ \\

\normalsize
These two parts can be thought of as a 'noise' and 'signal' term respectively.

In general, both $\phi$ and $\pi$ are unknown. Based on the excess of mutations we observed, we can calculate lower and upper bounds for $\pi$.

As an example, we can use the observations from McRae et. al, 2017 in which $968$ DNMs predicted to cause protein truncation were observed and $392$ were expected in a randomly ascertained population of size $N=4,293$ based on the trinucleotide mutation rate. In this case, we have:

$$ \frac{968-392}{4293} < \pi \leq \frac{968}{4293} $$

or

$$ 0.134 < \pi \leq  0.225 $$

The upper-bound holds for $\phi$ close to one, in which case the true number of expected mutations is zero, and all of the observed mutations are pathogenic. The lower bound cannot hold with equality except in the trivial case where there is no excess, so $\phi = 0, \pi = 0$. (right???)

For plausible values of $\phi$ (proportion of genes that are haploinsufficient and cause DD) based on constraint data (e.g. $\phi = [0.05 - 0.2]$),  $\pi \approx 0.14$.

\section{Applying the model to noncoding de novo mutations}

For the non-coding elements sequenced in DDD, we are interested in $\phi_{CNE}$ which is actually composed of two parts: $$\phi_{cne} = f_{elements} * f_{sites}$$ which is the fraction of elements and the fraction of sites within those elements that are pathogenic and contribute to DD risk.

If there are a subset of mutations with similar severity and penetrance in the CNEs as in the genes, then the following holds:
\[
\frac{\phi_{lof} * \mu_{lof}}{\phi_{cne} * \mu_{cne}} = \frac{\pi_{lof}}{\pi_{cne}} 
\]

We have:
$$\phi_{lof} = 0.10, \mu_{lof} = 0.09, \pi_{lof} = 0.14$$
$$\phi_{cne} = ?, \mu_{cne} = 0.03, \pi_{cne} = 0.03$$

So we estimate $\phi = 0.014$ and we can use the number of recurrently mutated elements we observed to constrain the likelihood of different combinations of $f_{elements}, f_{sites}$ such that $f_{elements} * f_{sites} = 0.014$

\section{Extending the model to recessive variants}

To consider recessive cases, we can write the model as:

\huge
$$E(v) = (1-\phi)N\lambda + \pi N $$ \\

\normalsize
Where $\lambda \propto \mu$ is the rate of biallelic variants in the target sites defined (e.g. genes).

As before, we have:
\[
\frac{\phi_{lof} * \mu_{lof}}{\phi_{rec} * \lambda_{rec}} = \frac{\pi_{lof}}{\pi_{rec}} 
\]

\section{Discussion}

There is clearly nothing left to discuss.

\end{document}